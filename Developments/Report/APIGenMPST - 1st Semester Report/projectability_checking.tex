\normalsize
This step is dedicated to analyse the provided \textbf{AST} in order to determine if it can be projected or not.\\
Projectability or wellformedness, is usually divided in a set of properties that must be maintained in order to comply with the demands stated in the Introduction and assure the overall well function of the protocols.\\
\\
\textbf{Projectability}\\
\\
\textbf{No self-communication:}\\
Forbidding self-communication was a trivially implementation.\\
Traversing the \textbf{AST}, every interaction of the type: \textbf{p$>$q:t} needs to be checked assuring that \textbf{p $\neq$ q}. \cite{cledou2022api}\\
\begin{lstlisting}[language=Scala, caption=No Self-Communication]
private def selfCommunication(global: Protocol): Boolean =
    global match
        // terminal cases //
        case Interaction(agentA, agentB, _) => agentA != agentB
        case RecursionCall(_)               => true
        // recursive cases //
        case RecursionFixedPoint(_, globalB) => selfCommunication(globalB)
        case Sequence(globalA, globalB)      => selfCommunication(globalA) && selfCommunication(globalB)
        case Parallel(globalA, globalB)      => selfCommunication(globalA) && selfCommunication(globalB)
        case Choice  (globalA, globalB)      => selfCommunication(globalA) && selfCommunication(globalB)
        // unexpected cases //
         case _ => throw new RuntimeException("\nExpected:\tGlobalType\nFound:\t\tLocalType")
end selfCommunication
\end{lstlisting}
\textbf{No free variables:}\\
Free recursion variables can not be present on the \textit{global type}. That is, a variable \textbf{X} in a \textit{global type} \textbf{G}, if previously there was a reduction of the \textit{global type}: \textbf{$\mu$X ; G}.\cite{denielou2011dynamic}\\
A special case is to treat recursion variables that appear in parallel branches as free variables, even when that same parallel was achieved in \textbf{G} after \textbf{$\mu$X ; G}. \cite{denielou2011dynamic}\\
Although the reason for generally forbidding free variables might be trivial, the special case may not be.\\
Through this action, we make sure that no race conditions arises from different iterations of the same loop.\\
\pagebreak
\begin{lstlisting}[language=Scala, caption=No Free Variables]
private def freeVariables(variables: Set[String], global: Protocol): Boolean =
    global match
        // terminal cases //
        case Interaction(_, _, _)    => true
        case RecursionCall(variable) => variables.contains(variable)
        // recursive cases //
        case RecursionFixedPoint(variable, globalB) => freeVariables(variables + variable, globalB)
        case Sequence(globalA, globalB)  =>
            globalA match
                case RecursionCall(variable) => freeVariables(variables, globalA) && freeVariables(variables - variable, globalB)
            case _                       => freeVariables(variables, globalA) && freeVariables(variables, globalB)
        case Parallel(globalA, globalB)  => freeVariables(Set(), globalA) && freeVariables(Set(), globalB)
        case Choice  (globalA, globalB)  => freeVariables(variables, globalA) && freeVariables(variables, globalB)
        // unexpected cases //
        case _ => throw new RuntimeException("\nExpected:\tGlobalType\nFound:\t\tLocalType")
end freeVariables
\end{lstlisting}
\textbf{Disambiguation:}\\
Disambiguation is concerned with the need of each role to determine what choice's branch was picked.\\
To meet this criteria it is needed that for each role, its first interaction in both branches are different, either in the roles involved, their actions or their datatype. \cite{denielou2011dynamic}\\
The current implementation gets the set of head interactions that a role has in each branch, then it checks if the intersection of those sets are empty, accepting if so or rejecting otherwise.\\
\begin{lstlisting}[language=Scala, caption=Disambiguation]
@tailrec
private def roleDisambiguation(globalA: Protocol, globalB: Protocol, roles: Set[String], isStillProjectable: Boolean): Boolean =
    if   roles.isEmpty
    then isStillProjectable
    else
        val role: String               = roles.head
        val headGlobalA: Set[Protocol] = headInteraction(globalA, role)
        val headGlobalB: Set[Protocol] = headInteraction(globalB, role)
        val isProjectable: Boolean     = (headGlobalA intersect headGlobalB).isEmpty
        roleDisambiguation(globalA, globalB, roles - role, isStillProjectable && isProjectable)
end roleDisambiguation

private def disambiguation(global: Protocol): Boolean =
    global match
        // terminal cases //
        case Interaction(_, _, _) => true
        case RecursionCall(_)     => true
        // recursive cases //
        case RecursionFixedPoint(_, globalB) => disambiguation(globalB)
        case Sequence(globalA, globalB)      => disambiguation(globalA) && disambiguation(globalB)
        case Parallel(globalA, globalB)      => disambiguation(globalA) && disambiguation(globalB)
        case   Choice(globalA, globalB)      => roleDisambiguation(globalA, globalB, roles(global), true)
        // unexpected cases //
        case _ => throw new RuntimeException("\nExpected:\tGlobalType\nFound:\t\tLocalType")
end disambiguation
\end{lstlisting}
\textbf{Linearity:}\\
The concept and therefore, the implementation of linearity is close to that of disambiguation.\\
For each parallel \textit{global type} we need to make sure that there are no similar interactions in both branches. \cite{cledou2022api}\\
We retrieve the set of all the interactions in both branches a role is involved in, then we  make sure that the intersection of both sets are empty, accepting if so or rejecting otherwise.\\
\begin{lstlisting}[language=Scala, caption=Linearity]
private def linearity(global: Protocol): Boolean =
    global match
        // terminal cases //
        case Interaction(_, _, _) => true
        case RecursionCall(_)     => true
        // recursive cases //
        case RecursionFixedPoint(_, globalB) => linearity(globalB)
        case Sequence(globalA, globalB)      => linearity(globalA) && linearity(globalB)
        case Parallel(globalA, globalB)      =>
            val iterationsA: Set[Protocol] = interactions(globalA)
            val iterationsB: Set[Protocol] = interactions(globalB)
            (iterationsA intersect iterationsB).isEmpty
        case Choice  (globalA, globalB)      => linearity(globalA) && linearity(globalB)
        // unexpected cases //
        case _ => throw new RuntimeException("\nExpected:\tGlobalType\nFound:\t\tLocalType")
end linearity    
\end{lstlisting}